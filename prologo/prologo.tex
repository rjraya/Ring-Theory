Querido lector. Tienes delante de ti el fruto de un trabajo confeccionado durante largas horas. Combina distintos enfoques y podrás notar que mis cualidades como redactor en latex fueron evolucionando a medida que avanzan los capítulos. Este es un regalo para que puedas cursar la asignatura de Álgebra I con mayor facilidad. 

También me gustaría explicarte que mi objetivo redactando estos apuntes no ha sido mostrarte mis cualidades como matemático sino más bien rellenar un vacío que es difícil de explicar. En este sentido, te recordaré si ello puede ser motivador para tí, las palabras que leí por primera vez en el texto \textit{Álgebra Lineal y Geometría I} del profesor Alfonso Romero: "Sé riguroso en tu percepción y no confundas la matemática con aquellos que te la muestran. Ella nunca te defraudará". 

Centrándonos en el ámbito del Álgebra Abstracta me permito referirte al prefacio del texto del profesor Robert B. Ash \textit{Abstract Algebra: The Basic Graduate Year}, uno de los pocos autores de textos para graduados que he visto expresarse con tanta sinceridad y verdadero interés en formar buenos matemáticos. Quizás, como en mí, provoque en tí un sano sentido crítico hacia la forma en que aprendes el Álgebra.

Finalmente, me gustaría agradecer a todos los que me han ayudado/aconsejado/orientado/inspirado en mis años de formación en la Universidad de Granada. Las presentes notas son fruto de su influencia.  

\medskip
\begin{flushleft}
  El Autor\\
  Granada, invierno de 2018\\
\end{flushleft}


